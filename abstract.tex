%definisco il layout dell'abstract
\def\changemargin#1#2{\list{}{\rightmargin#2\leftmargin#1}\item[]}
\let\endchangemargin=\endlist 

%Genero l'ambiente per l'abstract
\newcommand\summaryname{Sommario}
\newenvironment{Sommario}%
    {\begin{center}%
    \bfseries{\summaryname} \end{center}}
    
\begin{Sommario}
\begin{changemargin}{0.3cm}{0.3cm}
%Inserire abstract. I margini nell'abstract sono stati ridotti di un centimetro. In caso non si volesse questa riduzione rimuovere \textit{changemargin}.
Negli ultimi anni c'\`e stato uno sviluppo della tecnologia blockchain, che ha portato molti sviluppatori a creare nuove monete virtuali. Il processo di mining \`e il processo più importante che sta alla base del funzionamento di una criptovaluta, tramite il quale si creano nuove criptomonete e si confermano blocchi di transazioni.
In questa tesi si studiano i comportamenti statistici dei tempi di tale processo utilizzando la teoria delle code. Si approfondiranno due modelli di code e la trattazione si baserà anche sugli articoli presenti in letteratura. 
Lo studio della statistica dei tempi e degli aspetti che la influenzano, sono elementi importanti per capire quali sono i punti di forza e debolezza dei vari modelli. 
I risultati ottenuti, mostrano che si ottengono tempi bassi se la transazione ha una commissione alta o un pagamento elevato.
\end{changemargin}
\end{Sommario}
