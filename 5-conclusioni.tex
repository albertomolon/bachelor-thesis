In questa tesi, si sono volute analizzare le statistiche dei tempi di conferma delle transazioni e gli aspetti che le influenzano. Inoltre, si sono confrontati i risultati degli articoli presenti in letteratura per avere delle soluzioni più ampie. In particolare, si \`e visto che:
\begin{itemize}
\item per M/G/1, l'analisi risulta più completa considerando un \textit{batch service} anziche un servizio normale. Per questo modello, si sono studiati i fattori che impattano nei tempi di conferma e nei ritardi delle transazioni, concludendo che transazioni con alte commissioni o elevati importi, hanno dei tempi minori. Nel caso particolare di M/D/1, la durata del tempo di servizio \`e un compromesso dovuto alle possibili vulnerabilità che scaturiscono nel caso di tempi troppo lunghi o troppo corti;
\item per M/M/1, c'\`e il vantaggio di poter studiare con lo stesso modello anche il processo del consenso; inoltre, in termini di prestazioni temporali, \`e preferibile un'unica coda M/M/1 con tasso di servizio $\mu$ anzich\`e $m$ code con tasso $\frac{\mu}{m}$ ciascuna.
\end{itemize}
Lo studio delle criptovalute \`e tutt'ora un tema aperto, che può portare molti spunti interessanti per il futuro, soprattutto se un giorno, si decidesse di passare dai contanti alle criptomonete. La quasi totalità degli articoli presenti in letteratura, approfondisce soprattutto il processo di mining, che indubbiamente \`e il processo alla base delle monete digitali. Per studi futuri, si può pensare di ragionare anche sul processo del consenso, integrando lo studio con altri modelli di code.
