In questi anni, la tecnologia blockchain sta avendo grande considerazione da parte di aziende pubbliche e private per lo sviluppo di reti decentralizzate. Questa tecnologia \`e conosciuta soprattutto per essere alla base del funzionamento delle criptovalute. I nodi della rete si possono scambiare transazioni in modo sicuro grazie al lavoro di alcuni utenti, chiamati minatori, che attuano il processo di mining per confermare le transazioni e creare nuova valuta.\\
In questa tesi, viene studiato il comportamento statistico dei tempi di conferma delle transazioni, sfruttando due modelli di teoria delle code e confrontando i risultati presenti in letteratura.
Lo studio della statistica dei tempi e degli aspetti che li influenzano sono elementi importanti per capire quali sono i punti di forza e debolezza dei vari modelli. 

\section{Stato dell'arte}
Gli articoli presenti in letteratura, utilizzano modelli di code diversi adoperando approcci vari. Ad esempio, in~\cite{art:MM1}, si sfrutta il modello M/M/1 per il calcolo del tempo totale di sistema, visto come somma tra il tempo di mining e il tempo del consenso. Questo articolo, si basa a sua volta su~\cite{art4:GM1}, che \`e interessante perch\`e spiega dettagliatamente le ipotesi dei processi e fornisce una vasta panoramica dello stato dell'arte. Dopodich\`e, in~\cite{art2:MG1} e in~\cite{art3:MG^B1}, vengono studiati i fattori che impattano sul tempo di conferma di una transazione. I modelli utilizzati sono: M/G/1 per~\cite{art2:MG1} e M/$G^B$/1 per~\cite{art3:MG^B1}.\\
Inoltre, gli articoli~\cite{art:bc1},~\cite{art:bc2},~\cite{art:satoshi} e il libro~\cite{libro:bitcoin}, sono stati d'aiuto per la parte riguardante il funzionamento della blockchain e delle criptovalute.

%**************************************************************
\section{Organizzazione del testo}
La tesi \`e composta come di seguito:
\begin{description}
    \item[{\hyperref[cap:con-prel]{Il secondo capitolo}}] descrive le conoscenze preliminari per la comprensione del testo.
    \item[{\hyperref[cap:MG1]{Il terzo capitolo}}] approfondisce l'analisi del modello M/G/1.
    \item[{\hyperref[cap:MM1]{Il quarto capitolo}}] approfondisce l'analisi del modello M/M/1.
    \item[{\hyperref[cap:conclusioni]{Il quinto capitolo}}] illustra le conclusioni.
\end{description}
