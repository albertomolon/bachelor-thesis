\newfont{\ttvar}{cmvtt10 scaled 1200}     % nuovo carattere tipi courier a spaziatura variabile per le dimostrazioni

% margini senato
%\textwidth       =  15.45 cm            % larghezza 21 cm - 4 cm (sinistro) - 2.5 (destro)
\textwidth       =  16 cm 
\textheight      =  23.50 cm            % altezza 29.7 cm - 3 cm (superiore) - 2 (inferiore)
\topmargin       =   -0.20 cm            % margine superiore 3 cm diminuito di 1 inch
%\oddsidemargin   =   0.17 cm            % margine sinistro 4 cm diminuito di 1 inch
\oddsidemargin   =   -0.10 cm            % margine sinistro 4 cm diminuito di 1 inch

\evensidemargin  =  -0.04 cm            % margine destro 2.5 cm diminuito di un inch

\setlength{\headsep}{0.80cm}
\setlength{\footskip}{1.5cm}
\parindent = 0.7cm
\captionmargin = 0.7cm

%Interlinea 1.5
\onehalfspacing

%Funzione che mette numero_capitolo e nome_capitolo nella stessa linea
\makeatletter
\def\@makechapterhead#1{%
  \vspace*{0\p@}%
  {\parindent \z@ \raggedright \normalfont
    \ifnum \c@secnumdepth >\m@ne
      \if@mainmatter
        \Huge\sffamily\bfseries\space\thechapter.\space
      \fi
    \fi
    \interlinepenalty\@M
    \Huge \sffamily \bfseries #1\par\nobreak
    \vskip 40\p@
  }}
\makeatother

% stile pagina
\pagestyle{fancy}
\renewcommand{\chaptermark}[1]{\markboth{\chaptername\ \thechapter.\ #1 }{}}
\renewcommand{\sectionmark}[1]{\markright{\thesection\ #1}{}}
\fancyhead{}
%\fancyhead[LE,RO]{\sffamily \thepage}
\fancyhead[RE]{\sffamily \leftmark}
\fancyhead[LO]{\sffamily \rightmark}
\fancyfoot{}
\fancyfoot[C]{\sffamily \thepage}

% ridefinisco lo stile plain
\fancypagestyle{plain}{ \fancyhead{} \fancyfoot{}
\fancyfoot[C]{\sffamily \thepage}
\renewcommand{\headrulewidth}{0pt}}

% stile per i titoli
\allsectionsfont{\sffamily \raggedright}

% definisco i colori
\definecolor{codegreen}{rgb}{0,0.6,0}
\definecolor{codegray}{rgb}{0.5,0.5,0.5}
\definecolor{codepurple}{rgb}{0.58,0,0.82}
\definecolor{backcolour}{rgb}{0.95,0.95,0.92}
\definecolor{backcolourwhite}{rgb}{1,1,1}

%definisco stile listati di codice
\lstdefinestyle{mystyle}{
    backgroundcolor=\color{backcolour},   
    commentstyle=\color{codegreen},
    keywordstyle=\color{magenta},
    numberstyle=\tiny\color{codegray},
    stringstyle=\color{codepurple},
    basicstyle=\ttfamily\footnotesize,
    breakatwhitespace=false,         
    breaklines=true,                 
    captionpos=b,                    
    keepspaces=true,                 
    numbers=left,                    
    numbersep=5pt,                  
    showspaces=false,                
    showstringspaces=false,
    showtabs=false,                  
    tabsize=2
}

\lstset{style=mystyle}
\lstset{emph={RandomForestClassifier, RandomizedSearchCV, GridSearchCV},emphstyle=\underbar}
