%************
%* packages *
%************

%Codifica
\usepackage[utf8]{inputenc}

%Lingua, sostituire italian con english nel caso in cui la tesi sia scritta in inglese
\usepackage[italian]{babel}

%Titoli
\usepackage{titlesec}

%Pacchetto per definire layout di pagina
\usepackage{fancyhdr}
\usepackage{sectsty}
\usepackage[left=3cm, right=3cm, bottom=3cm]{geometry}

%Spazia linee all'interno del documento
\usepackage{setspace}

%Listati di codice
\usepackage{verbatim}
\usepackage{listings}

%Didascalie immagini
\usepackage[hang,small,sf,font=small, labelfont=bf]{caption}
\usepackage{subcaption}

%Inclusione immagini
\usepackage{graphicx}

%Impostazioni note a piè pagina
\usepackage[stable]{footmisc}

%Citazioni e riferimenti a label
\usepackage[english]{varioref}

%Colori
\usepackage[usenames]{color}
\usepackage{xcolor}
\usepackage{colortbl}

%Crea link ipertestuali
\usepackage[hidelinks]{hyperref}

\hypersetup{
            pdfauthor={Alberto Molon},
            pdftitle={Modelli di Code per la Descrizione del Processo di Mining di una Criptovaluta},
            pdfsubject={Information Security},
            pdfkeywords={Information Security, Teoria delle Code},
            pdfproducer={LaTex}
}

%Formattazione url
\usepackage{url}

%Inserimento formule
\usepackage{amsmath}
\usepackage{mathrsfs}
\usepackage{amssymb,amsthm}    % matematica
\theoremstyle{plain}
\newtheorem{teorema}{Teorema}

%Usato per diminuire spazio tra la parte superiore della pagina e l'indice
\usepackage{tocloft}

%Per avere tabelle più verso sx (soprattutto per quelle larghe)
\usepackage{changepage}

\usepackage{tabularx}

\usepackage[backend=biber,style=numeric,hyperref,backref]{biblatex}
                                        % eccellente pacchetto per la bibliografia; 
                                        % produce uno stile di citazione autore-anno; 
                                        % lo stile "numeric-comp" produce riferimenti numerici
                                        % per includerlo nel documento bisogna:
                                        % 1. compilare una prima volta Tesi.tex;
                                        % 2. eseguire da terminale: biber Tesi
                                        % 3. compilare ancora Tesi.tex.
\bibliography{bibliografia}

                                        
                                        

